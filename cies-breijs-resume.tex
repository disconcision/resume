%
% LaTeX source of my resume
% =========================
%
% Heavily commented to to fit even LaTeX beginners (hopefully).
%
% See the `README.md` file for more info.
%
% This file is licensed under the CC-NC-ND Creative Commons license.
%


% Start a document with the here given default font size and paper size.
\documentclass[10pt,a4paper]{article}

% Set the page margins.
\usepackage[a4paper,margin=0.75in]{geometry}

% Setup the language.
\usepackage[english]{babel}
\hyphenation{Some-long-word}

% Makes resume-specific commands available.
\usepackage{resume}




\begin{document}  % begin the content of the document
\sloppy  % this to relax whitespacing in favour of straight margins


% title on top of the document
\maintitle{Andrew Blinn}{}{\today}

\nobreakvspace{0.3em}  % add some page break averse vertical spacing

% \noindent prevents paragraph's first lines from indenting
% \mbox is used to obfuscate the email address
% \sbull is a spaced bullet
% \href well..
% \\ breaks the line into a new paragraph
\noindent\href{mailto:me@andrewblinn.com}{me\mbox{}@\mbox{}andrewblinn.com}\sbull
\textsmaller+1 (647) 909-6867\sbull
\href{http://github.com/disconcision}{github.com/disconcision}
\\
Toronto\sbull
Canada\sbull Spoken Languages : English + French (basic)

\spacedhrule{0.8em}{0em}  % a horizontal line with some vertical spacing before and after

% \roottitle{Summary}  % a root section title

\vspace{0.8em}  % some vertical spacing

 \emph{Passionate about programming languages as user interfaces; keeps current with PL/FP/UI research \& development }

\spacedhrule{0.8em}{-0.4em}

\roottitle{Work @ \href{http://todaqfinance.net/}{TodaQ, Toronto}}

  \headedsubsection
        {Software Engineer; Clojure}
    {May 2019 - Current}
    {\bodytext{Implementing a novel protocol for distributed digital asset management and its ancillary web services}}
    
\roottitle{Work @ \href{http://www.utoronto.ca}{University of Toronto}}

  \headedsubsection
    {Course materials development}
    {Summer 2018}
    {\bodytext{Developed code \& documentation for professor David Liu's programming language theory course. Designed \& implemented an educational language featuring pattern matching and algebraic data types}}
  \headedsubsection
    {Teaching assistant}
    {Sep 2017 -- May 2019}
    {\bodytext{Principles of Programming Languages (5 semesters); Introduction to Computational Thinking (1 semester). Responsibilities include lecturing, individual tutoring with a focus on TDD, code reviews, \& semi-automated testing. Built an algebraic code stepper to demonstrate continuations in Scheme}}


\spacedhrule{0.6em}{-0.4em}


\roottitle{Education @ \href{http://www.utoronto.ca}{University of Toronto}}

  \headedsubsection
    {H.BSc in Mathematics \& Computer Science}
    {2014 - 2019}
    {\bodytext{
    Led a student reading group on Category theory. Compilers coursework: Developed a $\lambda$-calculus-based language with a macro system compiling to x86 and transpiling to C. Coursework in abstract algebra including Galois Theory, differential geometry, topology, \& logic}}
  \headedsubsection
    {Research: Variational Data Structures with Marsha Chechik}
    {Sep 2018 - Current}
    {\bodytext {Developing \& profiling higher-order data structures in Haskell to underpin
    static analysis of software product lines. Developed \href{https://github.com/disconcision/spyshare}{SpyShare}, a Graphviz-based tool to visually introspect data  sharing}}
  \headedsubsection
    {Research: Structured Editing in Racket with Gary Baumgartner}
    {Summer 2017}
    {\bodytext{Self-initiated study of extant refactoring, live programming, and direct manipulation tooling, culminating in the design and implementation of a Racket-based polyglot structure editor; development ongoing}}


\spacedhrule{0.5em}{-0.4em}

\roottitle{Skills}

    \inlineheadsection
    {Functional Programming: }
    { Type- and Test-Driven-Development in Racket/Scheme, Clojure \& Haskell. \\ Property-based testing with QuickCheck. DSL development in Racket/Redex}

    \inlineheadsection
    {UI Design \& Graphics : }
    {CSS/HTML, mockups in Adobe CS incl. Photoshop, Illustrator, After Effects. \\
    Raytracing, raymarching, rasterization, \& kinematics in C++ \& GLSL}
    
    \inlineheadsection 
    {Performance Profiling \& Parallelism/Concurrency : }
    {C, C++: OpenMP, MPI; CUDA; core.async }
    
    \inlineheadsection 
    {Other languages \& tech : } 
    {\LaTeX, Emacs, Bash, Git, Java, Python, GNU/Linux/Windows/MacOS}
  


\spacedhrule{1.4em}{-0.4em}


\roottitle{Conferences \& Workshop Attendance}

  \headedsubsection
    {Clojure North 2019, Strange Loop 2018 , RacketCon 2018}
    {Various}
    
  \headedsubsection
    {ICFP 2018 (International Conference on Functional Programming)}
    {2018, St.Louis}
    {\bodytext{
    Attended fully-funded on PLMW Scholarship}}
    
  \headedsubsection
    {OPLSS 2018 (Oregon Programming Languages Summer School)}
    {2018, Eugene}
    {\bodytext{
    An intensive three-week program of lectures and workshops with leading PL researchers}}


\spacedhrule{0.6em}{-0.4em}

\roottitle{Personal Projects (see Github)}

\inlineheadsection
  {Projects:}
  {\href{https://github.com/disconcision/fructure}{Fructure}, a Racket-based structured editor focusing on composable refactoring, to be featured at \href{https://con.racket-lang.org/#speakers}{RacketCon 2019}. \href{https://github.com/disconcision/depthmarch}{Depthmarch}, A C++ raymarcher for constructive solid geometry, parallelized in OpenMP. \href{https://github.com/disconcision/containment-patterns}{Containment Patterns}, custom pattern matchers enabling concise updates of deeply-nested data structures}


\vspace{0.8em}

\spacedhrule{0.6em}{-0.4em}

\roottitle{Outside Interests}

\inlineheadsection
  {}{Year-round bike commuter. Running, bouldering, camping, photography}


\end{document}

＀
