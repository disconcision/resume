%
% Adapted by me from a template by Cies Breijs
% See the `README.md` file for more info.
% This file is licensed under the CC-NC-ND Creative Commons license.
%


% Start a document with the here given default font size and paper size.
\documentclass[10pt,a4paper]{article}

% Set the page margins.
\usepackage[a4paper,margin=0.75in]{geometry}

% Setup the language.
\usepackage[english]{babel}
\hyphenation{Some-long-word}

% Makes resume-specific commands available.
\usepackage{resume}

%\usepackage{fontspec}
%\setmainfont{Georgia} 
\begin{document}  % begin the content of the document
\sloppy  % this to relax whitespacing in favour of straight margins

% title on top of the document
\maintitle{Andrew Blinn}{}{\today}

\nobreakvspace{0.5em}  % add some page break averse vertical spacing

% \noindent prevents paragraph's first lines from indenting
% \mbox is used to obfuscate the email address
% \sbull is a spaced bullet
% \href well..
% \\ breaks the line into a new paragraph
\noindent
%   \textsmaller+1 (647) 909-6867\sbull
    \href{http://andrewblinn.com}{andrewblinn.com}\sbull
    \href{mailto:me@andrewblinn.com}{me\mbox{}@\mbox{}andrewblinn.com}\sbull
    \href{http://github.com/disconcision}{github.com/disconcision}
    

\textcolor{spacergray}{\spacedhrule{-0.27em}{0em}}


\roottitle{Research Interests}
\bodytext{
  Programming Languages \sbull
  Human-Computer Interaction \sbull
  Computing Education
  }
  
\roottitle{Education}
  \headedsubsection  {\href{https://rackham.umich.edu/}{University of Michigan} \sbull Ph.D Student, Computer Science}
  {September 2020 - Current}
  {\bodytext{
    Researching user interfaces for/as programming languages at Cyrus Omar's \href{http://fplab.mplse.org/}{FP Lab}.
    \\
    
    }}
    
  \headedsubsection  {\href{http://www.utoronto.ca}{University of Toronto} \sbull H.B.Sc in Mathematics \& Computer Science}
  {May 2019}
  {\bodytext{
    Graduate-level coursework in abstract algebra, compilers, graphics \& languages.
    \\
    Coursework in algorithms, concurrency, differential geometry, operating systems \& topology.
    \\
    Built a Racket-based x86/C compiler for a $\lambda$-calculus-based language with macro system.
    }}
    
\roottitle{Research Experience}    
  \headedsubsection {Techniques in Variability-aware Data Structures with Marsha Chechik}
  {2018 - 2019}
  {\bodytext{
    Built \& profiled Haskell data structures supporting variational analysis of software product lines.
    \\
    Designed \& built \href{https://github.com/disconcision/spyshare}{SpyShare}, a Graphviz-based tool to visually inspect data sharing.
    \\
    Created and modelled a system of GHC rewrite-rules using PLT Redex.
    \\
    \href{https://github.com/disconcision/vardatalab/blob/master/CSC495_TECHNIQUES_IN_VARIABILITY_AWARE_DATA_STRUCTURES.pdf}{Project Report} \sbull \href{https://github.com/disconcision/vardatalab/blob/master/CSC495_variational_data_structures_slides.pdf}{Presentation Slides} 
    }}
  \headedsubsection {Independent Study in Structured Editing in Racket with Gary Baumgartner}
    {Summer 2017}
    {\bodytext{Self-initiated study of existing refactoring, live programming \& direct manipulation tooling.
    \\
    Began work on \href{https://github.com/disconcision/fructure}{Fructure}, a Racket-based polyglot structure editor, and
    \href{https://github.com/disconcision/containment-patterns}{Containment Patterns}, \\ which extend pattern matching to capture contexts as composable continuations.
    }}
    
\roottitle{Conferences}
  \headedsubsection {Invited speaker at RacketCon}
    {2019 \sbull Salt Lake City}
    {\bodytext{
    Spoke about \href{https://github.com/disconcision/fructure}{Fructure}, a prototype structured editor focused on edit-time term-rewriting
    \\
    \href{https://www.youtube.com/watch?v=CnbVCNIh1NA}{Recorded Talk} \sbull
    \href{https://github.com/disconcision/fructure/blob/master/screenshots/REAL-RacketCon-Fructure-Talk.pdf}{Fructure Slides
    }}}
    
  \headedsubsection {Seat Filler}
    {Salt Lake City, Toronto, Eugene, St.Louis}
    {\bodytext{
    2019: \href{https://school.racket-lang.org/2019/plan/}{Racket's How to Design Languages Summer School}, \href{https://clojurenorth.com/}{Clojure North}. \\ 
    2018: \href{https://www.cs.uoregon.edu/research/summerschool/summer18/}{Oregon Programming Languages Summer School}, \href{https://conf.researchr.org/home/icfp-2018}{ICFP}, \href{https://www.thestrangeloop.com/2018/sessions.html}{Strange Loop}, \href{https://con.racket-lang.org/2018/}{RacketCon}}} 
    
\roottitle{Teaching}
  \headedsubsection{Course Development} {Summer 2018 \sbull University of Toronto}
  {\bodytext{
   Designed assignments and course materials for CSC324 - Principles of Programming Languages. \\
  Specified and built \href{https://github.com/disconcision/ductile}{Ductile}, a toy language demonstrating exhaustive pattern matching on ADTs. \\
  Implemented an \href{https://github.com/disconcision/racketlab/blob/master/choice-stepper.rkt}{algebraic stepper} to illustrate continuations and non-determinism in Scheme.}}
  \headedsubsection{Teaching Assistance} {University of Toronto}
  {\bodytext{
\begin{tabular}{ p{2.3cm} l{} }
  Winter 2019 & CSC324 Principles of Programming Languages\\
  Fall 2018 & CSC324 Principles of Programming Languages \\
  Fall 2018 & CSC104 Introduction to Computational Thinking \\
  Winter 2018 & CSC324 Principles of Programming Languages \\
  Fall 2017 & CSC324 Principles of Programming Languages
  \end{tabular}
  }}
 

\roottitle{Industry Experience}
  \headedsubsection {\href{http://todaqfinance.net/}{TODAQ Toronto} \sbull Software Development in Clojure}
    {May 2019 - August 2020}
    {\bodytext{On the back-end: Implementing and refining a new protocol for decentralized digital asset management based on a Merkel-trie-derived distributed data structure.
    On the front-end: Building user interfaces oriented around \href{https://andrewblinn.com/portfolio/todaq/}{reifying distributed digital assets}.}}

%\roottitle{Languages}
%\bodytext{
%  English \sbull French
%  }
\end{document}

＀